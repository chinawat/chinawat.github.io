%CIS260 Spring 2008
%LaTeX Tutorial Session Sample File
%Author: Chinawat Isradisaikul
%Thursday, February 21, 2008
\documentclass[12pt]{article} %specifies the type of the document.  The default font size is 10.  [12pt] is an optional argument to use font size 12 as default.
\usepackage{fullpage} %readjusts the margins
\usepackage{amssymb} %for special math symbols
\usepackage{mathtime} %Use Times font.  Comment this line if you run into problems; for example, if math symbols become weird.

%define new commands
%\newcommand{\command_name}[# of args]{definition} %Refer to argument i by #i.
\newcommand{\set}[1]{\ensuremath\{#1\}}
\newcommand{\R}[2]{\ensuremath #1\ R\ #2}

\begin{document}
Today is Thursday.  Tomorrow is Friday.  Yesterday was Wednesday.  There was total lunar eclipse last night, and I was out there taking pictures.  I took about 50 photos and I had a midterm today.  I didn't study for it, but I survived.  Yay!!!\\
Yahoo!!!\\ %\\ starts a new line without beginning a new paragraph
\LaTeX %This produces beautiful LaTeX symbol.
\today %prints today's date; changes when document rebuilt

Tomorrow we will have a quiz in recitation.

%The special characters in LaTeX are # $ % _ { } ^ \ ~.  To produce any of the first seven symbols, precede it by a backslash (\).
We know 100\% of \LaTeX\ today!!! %\ followed by a space produces a space in the document; otherwise we would have something like LaTeXtoday!!!  (See one with \today.)

``function'' %double quotation begins with `` and ends with ''

%center environment
\begin{center}
Combinatorial Proof
\end{center}

%math mode starts with $ and ends with $.  Expression goes between $'s.
%displaymath environment puts math expression at center to make it a little more attractive.  Its shorthand starts with \[ and end with \].
Let $x$ be a natural number.  Then $2x+1$ is odd.  \[1+2+3+4+5+6=21\]  $n \choose 2$
\[k {n\choose 2}=m {n+k\choose k}\] %things in {xxx} goes together.  Think about it as one large character.

$2\cdot 2=4$

\[\frac{13!}{5!8!}\]

%superscript ^; subscript _
$x^2-2x+1=0$ $2008^{260}$ $2^{2^2}$ $2^{2^{2^{2^2}}}$ $a_1$ $a_{12}$ $a_{1_2}$  Let $x_1$ and $x_2$ be the roots of this polynomial.

%ellipses are created by \dots, \cdots, \vdots, or \ddots
$1+2+3+\cdots+n=\frac{n(n+1)}{2}$ $a\mid b$

%tabular environment produces a table.  Need to specify the alignment of each column.  | puts a vertical line between corresponding columns.
%& splits text between columns.  A cell can be empty, i.e., && is allowed.
\begin{tabular}{lc|r} %align left, center, and right, respectively
First & Last & Points\\
\hline
Ant & Bat & 10\\
May & June & 7
\end{tabular}

%eqnarray environment enters math mode automatically.  There are three columns in this environment; the usage is similar to tabular.
\begin{eqnarray}
\frac{n(n+1)}{2}+(n+1)&=&\frac{n(n+1)+2(n+1)}{2}\\
&=&\frac{(n+1)[n+2]}{2}\\
\sum_{i=1}^{n+1}{i}&=&\frac{(n+1)(n+2)}{2}.
\end{eqnarray}

%eqnarray* environment suppresses equation numbers.
\begin{eqnarray*}
\frac{n(n+1)}{2}+(n+1)&=&\frac{n(n+1)+2(n+1)}{2}\\
&=&\frac{(n+1)[n+2]}{2}\\
\sum_{i=1}^{n+1}{i}&=&\frac{(n+1)(n+2)}{2}.
\end{eqnarray*}

%print the current page's number
This is page \thepage.

%enumerate environment creates an ordered list; can be nested up to four layers.  (Try 5 layers and see what happens.)
\begin{enumerate}
\item Determine whether the following relations...
\begin{enumerate}
\item $\neq$ on $\mathbb Z$.
\item $\subseteq$ on $2^A$, where $A$ is any nonempty set.
\item A relation $R$ on $\mathbb R$; $x R y$ iff $xy\leq 0$.
\end{enumerate}

\item
\item
\item
\end{enumerate}

%itemize environment creates an unordered list; same usage as enumerate
\begin{itemize}
\item Determine whether the following relations...
\begin{itemize}
\item $\neq$ on $\mathbb Z$.
\item $\subseteq$ on $2^A$, where $A$ is any nonempty set.
\item A relation $R$ on $\mathbb R$; $x R y$ iff $xy\leq 0$.
\end{itemize}

\item
\item
\item
\end{itemize}

This is page \thepage.

%\left and \right put a symbol tall enough to cover the inside expression
\[
\left(\frac{(n+1)}{2}\right]^3
\]

%example of half open interval
$[0,\infty)$

%array environment is similar to tabular, but used in math mode
\[
f(x)=\left\{
\begin{array}{ll}
\frac{x}{2} & \mbox{if $x$ is even}\\
3x+1 & \mbox{if $x$ is odd}.
\end{array}
\right. %\left must end with \right; \right. produces nothing
\]

%\noindent does not indent the new paragraph; \fbox produces a frame box.
\noindent\fbox{\bf CIS260--Spring 2008}\\
\textbf{CIS260--Spring 2008}

%usage of user-defined commands
$\set{x:x\in\mathbb Z}$ $\R{x}{y}$

\end{document}